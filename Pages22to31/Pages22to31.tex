
\subsection{As Dew from the Lord}

\begin{quote}
And the remnant of Jacob shall be in the midst of many peoples
like dew from the Lord,
like showers upon the grass,
which do not wait for anyone
and do not tarry for the children of men. Micah 5:6
\end{quote}

%FIXME: "fair" seems off.
This verse in the prophet Micah is a prophecy concerning the Church and its blessed work.
And it is not a mere dream of an enthusiastic visionary, a dream never become more than a shadow that vanishes. It is one of the divine promises that both has been fulfilled and that is continually being fulfilled until the day of Jesus Christ.

God willingly desires to fulfill this promise also in us and through us.

It is only a question of whether we are willing to receive the divine gift that is here promised also to us.

How can this come to pass? — Yes, thus asked Nicodemus that night when he was granted to speak with Jesus. And Jesus gave him a sharp answer, because Nicodemus ought to have known it; yet at the same time he also explained to Nicodemus the mystery of the Gospel of which there was question.

So might someone also answer us, when we ask: How can this come to pass? — that we ought to know better than to ask about that which all Christians ought to be acquainted with. But the Lord is good and gracious; he does not grow weary of explaining to us ignorant human beings his way and his will with us.

Micah’s prophecy concerning the remnant of Jacob, that it should be “in the midst of many peoples like dew from the LORD,” was fulfilled on that Pentecost day when the Spirit came upon Jesus’ disciples and made them into a congregation. Those praying disciples in the upper room were precisely “the remnant of Jacob.” Or upon whom could this name have fitted better than upon that small, trembling flock who, under distress and persecution, held fast to Israel’s hope, the promised and risen Messiah?

It is of these, and of those who were united with them in the same faith, that Paul writes in Romans chapter 11, verse 5: “So too at the present time there is a remnant, chosen by grace.”

Yes, precisely “chosen by grace”; for both the fact that there is a remnant at all is sheer grace, and every single one of these men and women, who together are called “the remnant of Jacob,” is among those who have been preserved by grace alone.

And precisely because they had no other hope and no other deliverance than grace alone, therefore there came upon them, “like dew from the LORD,” God’s own Spirit, who “does not wait for anyone and does not tarry for the children of men.” In the Lord’s own time, when he saw that “the night was far gone and the day was near,” then the dew fell from heaven and granted the longing hearts blessed refreshment. Then they rejoiced in the great works of God and praised him with a loud voice, because he had fulfilled his wondrous counsel of salvation, and had caused his own Son both to be born and to die and to rise from the dead and to ascend into heaven and to sit at the Father’s right hand, from where he had now sent them the Holy Spirit.

Truly, it was like dew from the LORD and like showers upon the grass. It is credible that there has never at any time been greater promise resting upon unlearned human hearts than precisely on that day; these truly came to experience what the times of refreshing from the LORD are for a human soul.

But precisely because dew from the LORD came upon these disciples and made them into a congregation, therefore they also became, in the midst of many peoples, like dew from the LORD.

They did not keep the blessing to themselves. They began to speak in other tongues, so that both the Jews who dwelt in Jerusalem and the devout men from every nation under heaven were compelled to marvel that they heard them speak, each in his own tongue, of the great works of God.


Thus those who themselves had been refreshed became a refreshment and a heavenly blessing to others. And thus the Church at once began to fulfill its task and to do its work upon the earth, to proclaim the Gospel of God in the midst of many peoples.

Even now the situation is the same, God’s gift is the same, and the Church’s task is the same. Even now the Lord wills that the remnant of Jacob should be in the midst of many peoples like a dew from the Lord and like showers upon the herbs. Even now he wills that it should not wait for anyone and not depend upon the children of men.

Have we received God’s gift, the heavenly dew? Have we experienced some Pentecost over our life, when our hearts burned within us with joy over the Lord’s great works? When the Spirit of God illumined Christ for the heart, so that this word about Jesus’ death and resurrection became something altogether new, because it became something for us, our personal salvation?

If this is so, then the Lord wills that we should present ourselves willingly for his service and for his holy warfare on the day of his power.

There are so many who wait for people, for this or that occasion, for this or that journey, for a powerful summons, for an urgent plea and exhortation, before they can bring themselves to take part. If there should arise some special prospect, or if there should come a man who can deliver a truly gripping address, then they will take part in the work for the kingdom of God. But this is not the true voluntary spirit. The Lord wills that his Church should be ready and willing because of the driving of the Holy Spirit, by virtue of the impulse that lies in this, that we are saved by grace, and that there are so many who know nothing of this salvation.

For even now God’s Church lives upon the earth in the midst of many peoples who do not know the Lord and his great works. And as long as this situation endures, the Lord wills that his Church should speak to the peoples in their own languages about the salvation that has come to it.

Consider what is meant by dew and rain. These are the names that the Lord gives to the Church in our text. Dew and rain do not exist for their own sake alone. They fall from heaven upon the dry earth, upon the fainting herbs, upon the withering grass. Where do they go? Before our eyes they vanish and pass away, the thousands and millions of drops. They sparkle for a moment in the sun and bear its image; then they disappear, and their place knows them no more. Yet their work is done, and life and flourishing are given to the earth and to its herbs.


Will the Lord truly will the same for us? Yes, indeed, he wills to make us as a blessing, if we are willing to let ourselves be used wholly and entirely, and exclusively, in his service. Nothing for ourselves, everything for him. He himself is our life; therefore he wills that we should live in such a way that we do not keep life for ourselves, but give, as he gave—then we also receive, as he received.

Therefore it is not possible for a Christian to work only occasionally, a little, for the Mission among the many peoples. This is our true task in life; it is our abiding calling. It is the goal of the congregation that the gospel should be preached to all peoples, to the end of the world.

Do we do this? Do all congregations do this? Yes, each may answer for himself. Most of us must surely confess that we do not live the life of the dewdrop, “in the midst of many peoples.”

Therefore it is time for us to awaken and to take to heart the Lord’s promise, which he wills to be fulfilled upon us and through us: “And the remnant of Jacob shall be in the midst of many peoples like dew from the Lord, like showers upon the grass, that do not wait for anyone and do not linger for the children of men.”

Gazeren, 1902, sider 113–116.


\subsection{Christ’s Coming}

\begin{quote}
Rejoice greatly, Daughter of Zion! Cry aloud, Daughter of Jerusalem! Behold, your King comes to you; righteous is he and full of salvation, lowly and riding upon a donkey, upon a colt, the foal of a donkey. And I will cut off the chariots from Ephraim and the horses from Jerusalem, and the battle bow shall be cut off; and he shall speak peace to the nations, and his dominion shall be from sea to sea and from the River to the ends of the earth. Zech. 9:9–10.
\end{quote}

The prophet sees in the Spirit Christ coming, and he portrays him as he sees him. He so earnestly desires that Zion, his royal city, and Israel, his people, should see him and know him and pay him homage. He wishes that no one should misunderstand the Messiah or take offense at him because he appears lowly and weak. Therefore he lays weight upon this: that the Messiah’s lowliness is in no way a sign of impotence or defeat; for victory and power shall nevertheless be his, even though he wins his victories in a manner different from the lords and kings of the world.

As Zechariah prophesied, so it also came to pass. Christ came, and he came in the greatest lowliness. He who at any moment could command legions of angels, he walked so patiently about among the people and allowed enemies to mock him, allowed unrighteous judges to condemn him, allowed scourgers to strike and soldiers to deride him; yes, he was as a worm and not a man, the reproach of men and the despised of the people. He was like a man in whose mouth there is no protest; like a lamb that is led to the slaughter, and like a sheep that is silent before those who shear it.

And yet even in his days of lowliness, in his deepest humiliation, there was nevertheless a power and fullness beneath the suffering, as though hidden behind the patience, which was felt in a strange way by all who had dealings with him.

Look at those who went out with Judas to seize him! How they immediately fell to the ground when he stepped toward them with his gentle word: It is I, Jesus of Nazareth; if you seek me, then let these go their way! And those priests who pronounced an unrighteous judgment upon him—how did their hearts not tremble! And Pilate, poor Pilate, how he writhed in distress over “this righteous man,” and how reluctantly he pronounced that Jesus should be crucified!

And they had good reason to grow pale and to tremble. For this righteous one whom they put to death proved himself by the resurrection from the dead to be the Lord of glory and the Prince of life. No wonder that secret fear already seized them while they sat in judgment over him who shall judge the living and the dead.

But at the same time that Jesus’ lowliness was an offense to many, because they expected a great outward glory in the Messiah, there were nevertheless also those who could confess: You are the Christ, the Son of the living God. They had found the Messiah in him, and they rejoiced over it with that unquenchable joy which accompanies the soul’s experience of the new, spiritual life.


They had known the quiet, mighty Voice that spoke peace to troubled hearts, and that brought storm and waves to be still. They had heard him who could say: Your sins are forgiven you! Fear not! Your faith has saved you! They knew that he was the one, true, rightful Messiah, and that no other was to be awaited.

But he who came to Zion and Jerusalem, poor and yet so full of salvation, so rich in heavenly love, he was not only to be a glory for Israel, he was also to be a light to the Gentiles, and his salvation was to reach to the ends of the earth.

Yet not by outward power was he to bring the nations under himself. He had no need of Ephraim’s chariots or Jerusalem’s horses in order to conquer the world. Rather, he first had to conquer the hearts of his own people before he could begin the conquest. Peter had to learn to put the sword into its sheath, and Paul had to learn to use the gospel of the cross as his weapon, before they could become servants of Christ to spread the kingdom.

It was by proclaiming peace to those who were near and to those who were far away that Jesus spread his kingdom among Jews and Gentiles. And while the kingdoms of the world rise and fall, the kingdom of Jesus Christ stands, and it grows and spreads under the enemy’s snorting wrath, and people after people pay homage to the King of peace.

He who came, still comes. For even yet his course of victory is not completed; even yet the gospel has not been proclaimed over all the earth. At times it seems to us as though it goes so slowly. We soon become discouraged, yes, even doubtful. We ask why he delays in coming, why the work of missions does not proceed much more swiftly, why the Lord does not drive more laborers out into his harvest, why he does not make his children more zealous for the cause of the kingdom.

But be still, you impatient heart! It is not in this way that you are to ask and complain. Indeed, you may complain over your faithlessness and sluggishness; but do not murmur against the Lord! Be assured that he does all things well, and he does not come too late to save.

No, wait only patiently, as you labor, until the day when a clear, radiant light dawns over all the Lord’s leadings. When you then are in the kingdom of light and see with undimmed eye the Lord’s course through time, then you shall surely also see this: that every people and every person has had the gospel sent to them at the proper time, and that none are unsaved among those who were willing to receive grace unto salvation.


Thus see: he comes—down through the changing ages—to one people after another. He speaks peace to the Gentiles and offers them a blessed entrance into his Kingdom.

And precisely now in our own days, when there is so much unbelief and ungodliness, sloth and cold indifference at home within Christendom, there is once again a glorious Advent season among the Gentiles. Christ comes and speaks peace to the Gentiles, and with swift steps the time draws near when the Gospel of the Kingdom has been preached to all peoples.

And what then? Then comes the End—the glorious outcome of the Lord’s dealings with the fallen race, the dreadful, final judgment over all those who have withstood the Kingdom of God unto the very last.

And what does all this mean for us, who in this last time confess faith in the reconciled and risen Christ Jesus, in the King who rules over all things?

Does it not mean this, friends, that we are called to labor while it is day? Does it not mean this, that “the King’s business requires haste”?

Behold, now is an acceptable time; behold, now is the day of salvation. Now the Lord comes, “poor and full of salvation,” with the Kingdom of grace for poor sinners, whom he calls out of darkness into his marvelous light. Soon he comes as Judge, with punishment for the resistant, with glory for his faithful ones.

Between this “now” and this “soon” lies our time for work. It is not an idle, chiliastic expectation, with hands folded in the lap, that is pleasing to the Lord. It is the self-sacrificing, self-denying love that seeks the lost and offers it rescue and salvation through faith in Jesus—this is what is required.

Soon the cry will sound for the last time: The Lord is coming! Then our day of reckoning also comes, when the interest on our pounds will be demanded. Let us therefore, in this Advent season as well, listen to the exhortation he has given us:

Trade until I come!\footnote{A reference to the Parable of the Minas (Luke 19:13). The Dano-Norwegian \textit{Kjøbslaaer} (rendered here as Trade'') emphasizes the servant's responsibility to actively utilize the spiritual gifts (the pounds'') entrusted to them by the Master until His return.}

(“Basieren,” 1905, pp. 370–373)


\subsection{From the rising of the sun even to its going down}

\begin{quote}
For from the rising of the sun and unto its going down my name shall be great among the Gentiles, and in every place incense shall be offered and a sacrifice brought to my name, a pure offering; for great shall my name be among the nations, says the Lord of hosts. Malachi 1:11.
\end{quote}

The writings of the prophets in the Old Testament contain many promises concerning the conversion and salvation of the Gentiles in the messianic age. Among these is also the above promise in the prophet Malachi. The Lord will show the Jews that their conceited and self-righteous thoughts concerning their own great worth, and the great worth of their worship in the sight of the Lord, were altogether vain and erroneous. Indeed, they were rather blasphemous toward God than pleasing to him. For the Jews imagined that, in truth, it was not they who ought to thank God, but that God in fact owed them thanks for their worship and their sacrifices; since there was, after all, no other people who worshiped and served the Lord except themselves alone. Therefore the Lord, they supposed, must be content, even if their sacrifices were not great and not so carefully in accordance with the Law; for a poor sacrifice was surely better than none at all.

Such thoughts were, as we have said, not merely conceited, arrogant, and self-righteous, but even blasphemous toward God. For the true God, the Creator of heaven and earth, the Lord to whom belong the cattle on a thousand hills, is surely not in need of Israel’s wretched sacrifices, as though he should desire them out of his own poverty and lack.

In order that Israel may now grasp and acknowledge its sin in this matter, and see how it in reality exalts itself and diminishes the Lord, the Lord therefore announces through the prophet Malachi the entrance of the Gentiles into the kingdom of God, and bears witness to how they shall bring to him true and right sacrifices.

Israel ought in truth to be ashamed of its delusions that the Lord cannot be without it and its sacrifices; for the Lord lacks no sacrifices. He is the God of all the earth; he is not the God of Jews only, but also of Gentiles; and he shall receive sacrifice and worship and honor and praise from the whole earth and from all peoples.


This is the Promise—the promise that is a cause of shame to the Jews, but to the Gentiles a promise that uplifts and consoles. Two things are chiefly contained in this promise:

1. that the Lord’s name shall become great, that is, known and spread abroad, honored and exalted from the rising of the sun to its setting among the Gentile nations; and

2. that, as a consequence of this greatness of God’s name, the Gentile nations shall bring to the Lord pleasing incense and pure offerings.

How is this to take place? Of this the Lord says nothing expressly here in this passage. But from the New Testament and from the history of the Gentile mission we know how the Lord intended to fulfill, and has fulfilled, this promise.

It sounds, to be sure, as though the Lord here proclaims that the ordinances and sacrifices of the old covenant were to be spread among all peoples, so that the Gentiles would become Jews in religion and worship, with the sole difference that they would be more faithful and more conscientious than the ancient Israelites were in Malachi’s time.

But the fulfillment shows us plainly that this is not how it was to come about, and that this is not how the fulfillment has come, insofar as it has come down to our own days on its way forward among lands and peoples.

God’s name has become great on earth among human beings through the salvation that is in Jesus Christ, and through the glorious Gospel concerning him and his saving work. That God has been manifest in the flesh; that God has sent forth his Son, born of a woman; and that he has suffered, died, and risen again for our salvation—this is the great wonder that has been preached among the nations and believed in the world, and that has made the Lord’s name great and exceedingly precious everywhere it has been heard. That God is our Father, the Son our Savior, the Holy Spirit our portion and our life—this is what makes God’s name great above all the Gentiles’ mute idols. For where is their revelation? Where is their love? Where is the salvation they have prepared?

When we compare the gods of the Gentiles and their cruel demands upon human beings with the true God and his blessed Gospel, the difference is surely felt, and then it is understood how great and precious, how joyful and praiseworthy the Gospel is. In this way the prophecy is meant, and in this way it has been fulfilled to this very day. Thus the Lord’s name has become great among the Gentiles, and thus it continues to become great from day to day.

The Lord, however, speaks not only of the fact that his Name, which formerly was known only in Israel, shall henceforth be known among all the nations; he also proclaims that he will receive “incense and a pure offering” from all the Gentile peoples.

What is meant by this? It is not meant that all peoples are to go to the temple in Jerusalem with animal sacrifices, with the blood of goats and calves. For the Lord himself has torn down the temple in Jerusalem and caused its sacrifices to cease long ago. No incense burns there; no animal’s blood is sprinkled upon the altar. Neither is it the Lord’s meaning that such temples as that in Jerusalem should be built throughout all lands; for he himself has said that he will be worshiped neither in Jerusalem’s nor in Gerizim’s temple, but in Spirit and truth.

And it is precisely this worship in Spirit and truth that is proclaimed here through the prophet Malachi, when it is said that incense shall be burned and a pure offering brought among all peoples. For the true incense before the Lord is the prayers of the saints (Rev. 5:8), and the pure offering from all those saved from among the Gentiles is their self-surrender to the Lord, when they “present their bodies as a living, holy, and God-pleasing sacrifice, which is their spiritual worship” (Rom. 12:1).

For where the Gospel of God has revealed the love of God, and where this love has been poured out into hearts through the Holy Spirit, there the true sacrificial fire burns, which consumes the old selfish nature and brings before the Lord the loving heart that gladly longs to do his good and perfect and holy will.

But this good will of God, which he desires to see fulfilled through the reborn and living human beings, includes precisely this as well: that his grace shall become known among ever more and more peoples upon the earth, until the knowledge of the Lord covers the earth as the waters cover the bottom of the sea.

For even yet the promise has not been fulfilled in its full extent; even yet not all Gentiles have become the Lord’s holy people, who bring him spiritual sacrifices. Even yet many burn their offerings upon the altars of idols; even yet fear and terror of false gods rest upon millions of human hearts.

But if the promise has not yet been fulfilled, what is it that holds it back? Nothing other than our sloth and sluggishness in the Lord’s service. We are indeed those saved from among the Gentiles who were to present ourselves as an offering pleasing to the Lord by being consumed in his service. We are indeed those who were to spread the Gospel, so that the Lord’s Name might be known among the Gentiles and praised among the peoples. We are indeed those who were to labor so that true incense and a pure offering might be brought before the Lord from the whole earth.

Are we active in this service, or is it for us a half-burdensome, half-indifferent matter, to which we devote little thought and still less prayer and labor? If we love the Lord, and if his love has freed us from the bondage of the world, then his will and his work are also precious to us; yes, we willingly enter his service and live on his errand. This is an important part of Christian freedom and of the work of the liberated congregation: to love those for whom Jesus died, and to bring them the liberating and saving Gospel. May this freedom and this service be united for us in the devotion of love and in labor for the advancement of God’s kingdom and the fulfillment of God’s promises!

“Basieren,” 1903, pp. 113–116.









