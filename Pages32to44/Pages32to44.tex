
\subsection{Fishers of Men}

\begin{quote}But as Jesus walked by the Sea of Galilee, he saw two brothers, Simon who is called Peter, and Andrew his brother, casting a net into the sea; for they were fishermen. And he said to them: Follow me, and I will make you fishers of men. Matt. 4:18–19.
\end{quote}

These verses tell of something among the very first that Jesus did after he had begun his public ministry. They have always drawn much attention, both from believers and from unbelievers. Unbelief has mocked him who would found the Kingdom of Heaven upon earth—one who seemed to possess neither greater understanding nor greater success, in that he called fishermen to help lay the foundation of so great a building.

Faith, however, has rejoiced and continues still to rejoice over that for which Jesus gives thanks to the Father: “I thank you, Father, Lord of heaven and earth, that you have hidden these things from the wise and understanding and have revealed them to the unlearned;” the same which Paul expresses in this way: “Not many wise according to the flesh, not many mighty, not many of noble birth are called; but what is foolish in the world God has chosen to shame the wise, and what is weak in the world God has chosen to shame the strong, and what is low in the world, and what is despised, and what is nothing, God has chosen, in order to bring to nothing that which is something, so that no flesh may boast before him.”

For if, on the one hand, it is true that there could scarcely have been found more insignificant people to serve as founders of the Kingdom than Galilean fishermen, then, on the other hand, it is altogether clear that if these Galilean fishermen could become instruments for the great work which may be seen on this day, this is all the more to the glory of God and to the praise of our Lord Jesus Christ, since it is a proof that the surpassing power which was granted to these fishermen is not from human beings, but from God.

And as all the Word of God, so also these verses are written for our learning; for they do not merely recount a bygone event which now has only historical interest, but they lay down the governing principle for the work of the Kingdom of God for all times. For what Jesus willed then, when he called Simon and Andrew, he wills still. And just as it took place then that a fisherman could become a fisher of men, so it takes place still:

“Follow me, and I will make you fishers of men!”

The following of Jesus is the first thing.

If the unbelieving mockers had understood all that lies therein, they would scarcely have despised the men whom Jesus called to help build the Kingdom of God upon the earth.

“Follow me.” Oh, what a call and what a task! Look upon him who went before, behold his way and his walk, behold his struggle and victory, his work and suffering, his humiliation and his glory, and judge for yourself whether it is such a small matter to follow Jesus that the experience and testimony of those who have followed him are of no worth!

Or test for yourself what it is to take up the cross and follow Jesus, and you will surely perceive that the demand is so great, and according to natural feeling and understanding so beyond human measure, that already concerning the fulfillment of this first demand Jesus says: For human beings this is impossible, but for God nothing is impossible.

Therefore, if any person ever truly becomes a follower of Jesus, one who leaves all things, denies self, renounces the world, takes up the cross and walks in the footsteps of Jesus through tribulations unto glory, then in this very fact the power of God is revealed in such a manner that one no longer marvels so greatly at the powers for work that unfold themselves in such a person’s life.

But is there any such true follower of Jesus among people now? Yes—if there is any true Christian, or anyone who truly shall be saved; for no one is saved except the one who walks the way of the cross unto salvation. And surely there are still some—also in our own days—who set their course toward heaven?

“Follow me,” says Jesus, “and I will make you a fisher of men.” Take note of his words; he does not say: follow me, and you will become a fisher of men; but rather: I will make you a fisher of men. Even for a follower of Jesus it does not happen of itself that one becomes a fisher of men; Jesus himself must make his followers into fishers of men.

And—God be praised—he does so. He gives his disciples this inward urging in their hearts, that they may have people saved, that they may bring more along with them on the way to heaven—more and always more; soul after soul must be given a share in the joy in God, for it is great enough for all; there is room, and again room, for guests at the Lord’s great supper, and again and again the Lord sends out his servants after more.

And Jesus does not merely give the desire to win souls; he also shows his disciples how it is to be done. He himself places the gracious invitation upon the tongues of his servants: Come, for now all things are prepared; come to the wedding feast! Go out into the streets and lanes of the city and gather them in! Go out to the crossroads and along the hedges and press them to come in, that my house may be filled!

Yes, that was indeed how it was in the old days, you think—but now, now surely there is no such commission for us? It was the apostles who were to be fishers of men, not we?

Well yes, it appears as though the power of Christianity has come to an end, and as though there are not many who wish to be fishers of men; yet the Lord’s power is still the same, and his Spirit is mighty as in former days, and even now he calls followers whom he will make into fishers of men.

And we cannot hide behind the apostles and say: they must go, but not I; nor behind the pastors and say: they must go, but not I. For who are the heirs of the apostles if not the congregation that is built upon their foundation? And whom does Paul exhort when he says, Be followers of me, even as I am of Christ, if not all true members of the congregation? Truly, the Lord still wills to have the members of his body engaged in the same labor as in the ancient days. And he continues to call his followers, his whole congregation and each of its individual members, to be fishers of men.

And the sea is just as deep and full of fish, and the world just as great and full of people. And though the congregation has grown and the number of Christians increased, yet there are perhaps just as many heathen left to preach to now as in the days of the apostles. For the number of the heathen has also grown.

And the doors are open and the gates burst apart, and the ways are prepared to all lands and all peoples. And as the days pass, the call of Jesus sounds ever clearer and more earnestly to all who hear his word: Follow me, and I will make you fishers of men!

Basieren, 1904, pp. 33–35.

\subsection{The Salt of the Earth}

\begin{quote}
You are the salt of the earth; but if the salt loses its power, with what shall it be salted? It is no longer good for anything, except to be thrown out and trampled underfoot by men. Matthew 5:13.
\end{quote}

Jesus prayed for his disciples in this manner: I do not pray that you should take them out of the world, but that you should keep them from the evil. This prayer of Jesus concerns precisely the same matter as his testimony in the Sermon on the Mount: You are the salt of the earth—namely, the serious calling which the Lord has given to his believers, the responsible position in which he has placed them in the world, a position that entails such great dangers for themselves that only by the grace of God can they be preserved, so that they themselves may be saved.

The salt of the earth is the name given to believers in contrast to the corruption which, through sin, is in the world. Yet this corruption does not consist precisely in coarse vices; rather, it consists first and foremost in this, that earthly, sensuous, and perishable things are the measure of human imagining and desiring. Paul describes this corruption, or rottenness, in this way: Many walk, of whom I have often told you, and now tell you even with tears, that they are enemies of the cross of Christ; whose end is destruction, whose god is the belly, and whose glory is in their shame, who set their mind on earthly things.

See, this is precisely the matter. Spiritual death, which is separation from God, has spiritual corruption as its immediate consequence. But spiritual death is found exactly where the soul desires only earthly things and the satisfaction of the senses.

But this is merely another name for paganism; for Jesus himself says: after all such things the Gentiles seek—namely food and drink and clothing; for them this is the whole goal of life, nothing else has value for them.

It is told of a guest who was asked whether he knew the Lord’s Prayer, that at first he answered no, saying that he had forgotten it; but when the missionary asked whether he remembered nothing of it at all, he finally broke out: Give us today our daily bread! That he had not forgotten. And this is characteristic of the great multitude of pagans: they have sense for food and drink, but what dullness, indifference, and incapacity to receive, when it comes to questions concerning that which is spiritual.

When therefore Jesus says of his disciples: You are the salt of the earth, he thereby designates a high and holy task, a calling which he gives them, for which they are by no means fit of themselves or by their own fleshly nature. For by nature they were children of the world, just like all the others.

It is only life that helps against death; it is only spiritual life that helps against spiritual death. It is only by passing over from death to life that anyone becomes salt in the sense of which Jesus here speaks. There can be no question of rescuing or preserving the spiritually dead from corruption except through life itself. When therefore the word sounds forth to those who in our day are called Christians, it becomes, surely, for all who still have something of genuine seriousness left in their souls, a sharply testing word—this: You are the salt of the earth.

Are we so? Yes—are we truly so? Have we passed from Death to Life? Have we come out of that corruption which is in the world through lust, and entered into that life of love which is wrought by God’s Spirit, and which in itself is mighty to endure eternally and has power to draw others with it out of Death and into Life?

For surely all who are willing to see can see that many bear the name of Christ and yet are nevertheless enemies of the cross of Christ, and with all their might help to promote spiritual death and rottenness among human beings.

But we do not merely call ourselves Christians in the sense that we live in a Christian land; we call ourselves a congregation and children of God, holy and beloved, chosen and precious to God. Are we then also salt? Is it felt and does it show in our life that we stand in the way of Sin and Death? Is a power of Life recognized in our speech and in our conduct? Is it as the apostle says: Let your speech always be gracious, seasoned with salt, so that you may know how you ought to answer each one? Do we walk wisely toward those who are outside, so that we redeem the opportune time?

Truly, it becomes a hard self-examination for the congregation when it must come forth into the light of the Word and ask itself: Am I Christ’s body? Do I live his life, and do I do his work? Yet this is indeed the meaning of this word: You are the salt of the earth.

But—God be praised—even if they are ever so few, there are nevertheless some who have passed from Death to Life, and who stand manfully in the work and the struggle for Life.

Yet even then this question arises: What do you salt? How far does your influence reach? Jesus says: the salt of the earth—the earth. So far must the influence of the congregation reach. It does not stop at family or circle of friends, nor at the boundaries of the local congregation, nor at the fatherland or our own nation. Further onward and farther out the calling goes. As far as Death goes, Life must attempt to set things right. Is that not the meaning?

But where do Sin and Death reign more undisturbed, more absolutely, than out in heathendom? And you believe in him who through death rendered powerless the one who had the power of death—that is, the devil—and who thereby intended to set free those who through fear of death were held in bondage all their lifetime. You believe in the Overcomer of Death; you believe that the gospel of Jesus is a healing remedy against Death.

What do you do then? Do you bring the Gospel to the dying or to the dead? See, this is precisely what is required: that the salt comes into the closest possible contact with that which is to be salted. Why then do you keep yourself so far away? Why do you stand at a distance?

The earth receives no benefit from the salt unless it is used, unless it is given over, surrendered, sacrificed. If it is a great task, then surely it is also a painful task. It is as Paul says: We who live are continually given over to death for Jesus’ sake, so that the life of Jesus may also be made manifest in our mortal flesh.

This is the nature of the salt. It is used up; it is worn away; yet it preserves and rescues and saves life, and it has accomplished its work.

But if you do not dare to enter the struggle of life against death, then the danger is great. If the Church does not dare to take up its calling and its cross, then its own life withers and languishes. And if it aligns itself with the world, then it falls under the same judgment and corruption as the world.

Therefore this is the condition of life for Jesus’ disciples and his Church: Truly, truly I say to you: Unless the grain of wheat falls into the earth and dies, it remains alone; but if it dies, it bears much fruit. Whoever loves his life will lose it, and whoever hates his life in this world will keep it for eternal life.

Let us therefore no longer fear being consumed in the Lord’s service and in the labor of his love. He who himself gave his life over to death and was gloriously raised from it—he will also grant to those who are given over to death for his sake a glorious resurrection and a blessed entrance into eternal life.

“Basieren,” 1904, pp. 49–52.


\subsection{Treasure Collectors}

\begin{quote}
Lay up for yourselves treasures in heaven, where neither moth nor rust corrupts, and where thieves do not break through nor steal.  Matthew 6:20.
\end{quote}

Guard your heart, says the wise man, above all else that you guard; for from it life proceeds (Proverbs 4:23). And Jesus also gives us to understand the worth of the heart, when he counsels us not to gather treasures on earth, but in heaven, because “where your treasure is, there your heart will be also.”

Earthly treasures are so dangerous and perishable because they draw the heart downward and bind it to this earth and to the present world.

Heavenly treasures have so beneficent an effect, because they help to make us heavenly-minded and to lift the heart toward the unseen and eternal things.

And this may help us see the matter still more clearly when we read John’s exhortation in his First Epistle: Do not love the world, nor the things that are in the world. If anyone loves the world, the love of the Father is not in him. For all that is in the world—the desire of the flesh, and the lust of the eyes, and the pride of life—is not from the Father, but from the world. And the world passes away, and its desire; but the one who does the will of God abides for eternity.

That which the heart loves and clings to occupies the heart day and night. Thought and desire always turn back to it, when for a brief while one has been compelled to think of something else.

They are strong bonds that bind the miser’s heart to his wealth, or the ambitious and vain to his greatness, or the pleasure-loving to his enjoyments. It is all the same what may for a moment disturb those who love the world in their love and enjoyment: involuntarily the heart nevertheless turns back to the beloved object and dwells upon it. In labor and in rest, by day and by night, at home or in church, the heart nonetheless always returns to its treasure.

And so it is likewise with the one whose heart is bound to the Lord with eternal love. Even if the world intrudes and disturbs, and even if labor and daily toil demand attention, in the innermost depth of the heart the direction remains the same. And as the magnetic needle points toward the pole, so the heavenly-minded heart always turns back to God and his eternal love, because only there is blessedness.

And by this are life and death determined. Think of the rich man and Lazarus! The love of the heart determined their eternal and unchangeable destiny.

But Jesus wills that his disciples should not only attain heaven, but that they should also find treasures there—treasures which they have gathered during their life on earth. What can he mean by this? Is it not the same thing he says elsewhere: Make for yourselves friends by means of the unrighteous mammon, so that, when you depart from here, they may receive you into the eternal dwellings?

It will surely not be all the treasures, nor the whole treasure; yet it is certain that it will be a glorious treasure, one that will gladden one for all eternity, if there are found friends in the world to come who are bound to us in this present world through the deeds, small or great, which we were able to show one another here below by means of earthly resources.

And Jesus, who wills that his disciples should have all the joy and blessedness that human hearts are capable of holding, therefore counsels them precisely for this reason to gather for themselves treasures in heaven—such treasures as consist in the joyful gratitude of redeemed people.

And here too the mission to the heathen has its place. It is one of the ways in which the unrighteous mammon can be used to gather treasures in heaven.

It will indeed be a true heavenly joy of the right kind, when we are granted the sight of the redeemed entering into the kingdom of glory, and when they are gathered from north and south, from east and west, to the marriage supper of the Lamb and the heavenly feast.

To see them—multitude after multitude—come forward to their place before the throne, and among them to recognize the friends we had gained, and the souls we had won, while we were on earth—how glorious and uplifting this will be!

And when the redeemed from among the heathen come forward, then to meet those for whom we prayed, and those for whom we sacrificed—will not that be worth all the sacrifice that it cost us to send them the Savior’s gospel? Indeed, no one will regret that he gathered such treasures in heaven. The possessions we leave behind on earth are those which were not sacrificed; the possessions we find again, a thousandfold refined and transfigured in heaven, are precisely those which were sacrificed for the cause of God and of his kingdom. For in this it is indeed true that we lose what we keep; but we gain for eternity what we sacrifice.

But how do we become diligent and good gatherers of treasure? Yes, we can learn that from those who gather treasures on earth. They do not ask, How little can I get by with? but they ask, How much—oh how much—can I lay aside in the treasury? How much can I put together this week, this month, or this year? In the same way must Christians also think about their treasures in heaven: not, How little can I get by with, but, How much can I lay up. That is true zeal in the gathering of treasure.

If we saw it in this light, how utterly different the question of giving to the mission among the heathen would appear to us! With a joy infinitely greater and purer than that of the miser over his growing treasures, we would then take part in bringing forth our gifts for the work of the salvation of the heathen. With steadily increasing interest we would give heed to every sign of the work’s progress and to the increase in the number of those who are saved.

Life would come into prayer and labor and gifts, if we were permitted to see that in this way our heavenly treasures increase, as the work advances and souls are won through the apostolic work in which we share; and we would soon forget that we had offered and renounced anything, when in truth we had only exchanged earthly possessions for eternal treasures in the Kingdom of God.

Lay up for yourselves treasures in heaven. This is Jesus’ own counsel and his own command to those who would belong to the Kingdom of God and become partakers of its glory. Let us follow his word; only in this way can we put his power to the test. Let us take our gifts and abilities, our time and our life, our goods and possessions, and lay it all down before him and say: I am not my own, but I am yours; take me into your service, and use me in your work, and teach me to employ all that I have for the advancement of your Kingdom in the manner that you yourself will.

If we could succeed in being so wholly devoted to Jesus and to have the eternal Kingdom of God before our eyes in all things, then we would be free and happy by reason of love, which is the heart’s one true good portion, which through ages upon ages shall not be taken from us. And what we gave from the impulse of love for the work of the Kingdom of God would be repaid to us many times over in the resurrection of the dead, when the friends whom we won would greet us and welcome us into the eternal dwellings.

Source: “Basieren,” 1904, pp. 81–84.

\subsection{Mission Prayer}

\begin{quote}
Ask, and it shall be given you! Matt. 7:7.
\end{quote}

How good it is to hear the words of Jesus, for no human being has ever spoken like this. He spoke as one who had authority, and not as the scribes. And precisely this authority—how comforting it is! What Jesus says stands so firm, and gives the soul such assurance and peace. And especially when his promises come to us in troubled hours, when there is “conflict without and fear within,” then this calm and elevation of his become a refreshing rest for the soul and work in the same manner as his mighty word to the storm and the raging waves: Peace, be still.

So it is also with this promise concerning answered prayer: Ask, and it shall be given you! No human being speaks of prayer in this way; for no one knows the Father’s mind and the Father’s power as the Son does.

And we need it deeply in connection with our missionary work. For when we look at the many difficulties that place themselves in the way of this “holy war” against Satan’s dominions and the power of darkness, then we feel deeply our own impotence and worthlessness; and despondency settles upon the soul, so that we are almost ready to give up everything, because we are fit for nothing.

Therefore missionary prayer is for us such a vital matter that we can scarcely exercise ourselves enough in it, and become both more capable and more zealous in it from day to day.

And now it is, of course, self-evident to all spiritually enlightened Christians that the sum of all that we ask according to God’s good and pleasing will is the prayer for the Holy Spirit. Jesus himself has made it entirely clear to his disciples that the Holy Spirit is the best gift, the true gift, the gift above all gifts for the children of God. Thus missionary prayer also is, in its inmost essence, precisely a prayer for God’s Holy Spirit. For he is the one who raises up missionaries, who drives them out into the work, who upholds them in the task and gives them fruit from their labor; he is also the one who fills the congregation with zeal and holy ardor in this cause, so that it gives itself no rest until it has sent forth the Gospel of the Kingdom to all peoples.

But this does not mean that, either in mission prayer or in any other prayer, we should mention only this one word—the Spirit—and then let all thought and care fall away. God is well pleased to hear his children’s many sighs and to listen to their many prayers.

The mission prayer comprises several different workings of the Spirit; it also asks for the Spirit’s gift for more than a single person. When we pray for mission, then it is surely one of the first things we pray for that the Lord of the harvest will thrust workers into his great harvest. In other words, the mission prayer is first 

\medskip
\begin{center}
\textsc{A Prayer for Missionaries}
\end{center}
\medskip

Missionaries who are obtained without prayer are not of the right kind. First and foremost, the one who is to become a missionary must be driven and led by the Spirit of God to pray with Isaiah: Here am I; send me! It often goes so that the Lord’s first call to the one who is to become a missionary is met with Moses’ fearful prayer: Lord, send another! But it does not become true earnestness until the fire from the altar has burned away both sin and fear…, so that the one who has himself experienced the great grace that his sin is taken away is also freed from fear, so that he both must and dares to pray: Send me! send me!

But not that alone. All who wish to have a missionary to send, every congregation that wishes to step into the ranks of the mission-sending congregations, must follow in the footsteps of the congregation in Antioch, which received Paul and Barnabas as its missionaries through prayer and fasting (Acts 13).

Just as only that one becomes a true missionary who prays: Lord, send me! so only that becomes a true mission-sending congregation which prays: Lord, use me and what is mine in your service, so that I may have a part in sending out missionaries. Give us men to send and hearts to send them!

But not everything is done by sending out missionaries. Think if we only sent them and did not pray for them! That would become a dangerous mission. Therefore the mission prayer is also

\medskip
\begin{center}
\textsc{A Prayer for the Missionaries}
\end{center}
\medskip

Those who are sent out with God’s Gospel to the heathen have gone out for us, in our stead, into the great conflict for the Kingdom of God; how then should we not support them? Can a people forget the soldiers who risk their lives in the struggle against the enemies of the people—does it not bear them and uphold them in every way? So neither can the children of God leave the solitary men and women who labor as light in the darkness of heathendom standing forsaken in their work and their struggle. Let us uphold their hands—yes, their very souls, their courage, their strength, their willingness, and their self-surrender—through our believing prayers.

These our envoys should not have to spend their strength alone; nor should they be alone in treading the winepress, so that they would have to lament that there was no one who helped, and none who supported. No, these champions in the war of light against darkness should feel and perceive that there stands behind them a great host of intercessors who, with sympathy and love, follow them upon their toilsome path. This would hold up their hands and preserve their boldness. Therefore it would also grant them fruit from their labor, so that they themselves might be preserved from falling under the power of darkness, and that heathen might be delivered from Satan’s bonds and chains and be led into the Kingdom of God and into the fellowship that is in the light with the Father and with his Son, Jesus Christ.

To this prayer for the missionaries and their work and its rich fruit among the heathen there belongs

\medskip
\begin{center}
\textsc{Prayer for Means}
\end{center}
\medskip

for the carrying out of the heathen mission. For it costs something to carry on heathen mission, and surely there is no one who thinks that our envoys should bear the expenses alone? Who serves as a soldier at his own expense? Or should the congregation of God choose as missionaries only such men and women as themselves possess the means to sustain both themselves and their work? No, the means for the missionaries’ support—for churches and schools, for houses and textbooks—where shall they come from if not from God through the voluntary hand of his children? Let us therefore pray that God would give means for all our mission work; and when we thus pray, let us consider that in truth we are praying to receive ourselves the blessed gift of sacrificial willingness. The true mission prayer therefore becomes also

\medskip
\begin{center}
\textsc{Prayer for Congregation}
\end{center}
\medskip

for congregation is precisely that body of Christ on earth which, through the Spirit, is united with the Head, Christ, and therefore also equipped with the power of Christ and the love of Christ, so that it with both joy and strength does his work and goes on his errand.

There is first real earnestness in mission work when we truly become congregation—that is, when Christ pours out his Spirit upon us—so that it becomes as on that first Pentecost day, when the Spirit came and tongues were loosed, and they all spoke of the mighty works of God in the various languages.

And even if we must pray for many things in our mission prayer, we nevertheless conclude where we began and say: all mission prayer is indeed, in truth, prayer for the Holy Spirit. If only it might become truly living among us, then we would receive both men and means and rich fruit of the work, both abroad and at home.

Ask, and it shall be given to you.

“Basieren,” 1904, pp. 177–180.














