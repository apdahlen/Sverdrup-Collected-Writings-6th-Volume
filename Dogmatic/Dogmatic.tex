
\section*{Dogmatic Paragraphs on the Congregation}

§ 1. To treat the doctrine of the Congregation after the doctrine of the Means of Grace, in such a way that the Congregation would be “the society of men among whom the Means of Grace have had their intended effect” (Communio), does not correspond to the Trinitarian recognition of the coming-into-being of the Congregation, nor does it accord with the individual Christian’s experience of how he himself has been incorporated into the Congregation.

§ 2. The Congregation came into being through a divine act which is part of God’s revelation among men and belongs to the eternal counsel of salvation, which had in view the restoration of the human race from its fall. This divine act consists in the outpouring of the Holy Spirit\footnote{In the authors’ manuscript there stands “sending forth,” which may perhaps be a scribal error. — Ed.} into the hearts of Jesus’ disciples at the first (Jewish) Feast of Pentecost after Christ’s Ascension. The signs which accompanied the outpouring of the Spirit are such as reveal the nature of the Spirit’s activity and what it proclaims. The storm signifies the mighty power; the tongues of fire, the heavenly light; the foreign tongues, the spread among all peoples.

§ 3. The divine visitation which the outpouring of the Spirit brings upon Jesus’ believing disciples is, according to Jesus’ promise in the Gospel of John, chapters 14–17, the Gospel of Luke 24:49, and Acts 1:4–8, and according to the Trinitarian consciousness of the Spirit’s operation: new, divine life, and new, spiritual light, joined with the corresponding power for action and impulse to bear witness. Grace—the divine life which in its essence is love—and the spiritual light which is truth, proceed from the exalted and transfigured Jesus Christ, with whom the Spirit unites the believing disciples, so that they become His body, and He their Head.

§ 4. The Congregation is therefore, for Trinitarian recognition, a new creation, a new race which once more bears the image of God; yet not a new bodily human race alongside the old, but a renewal of the old race through the forgiveness of sins and the life of the Holy Spirit in the hearts, so that this society lives the renewed human life within the outward conditions of the fallen race, surrounded by sin in the flesh and in the world, and with temporal death as the regular, though not unavoidable, exit from the present world.

§ 5. The congregation, which consists of those who have received the Holy Spirit, consolidates its outward form, reveals itself, and shows itself active and capable of inner and outer growth through the use of the gifts of grace and the means of grace. All have received the same Spirit, but not therefore precisely the same gifts; and thus there arise several different spheres of activity in the congregation through such persons as God has endowed with gifts, and whom the congregation has received through its recognition of those gifts. Of all activity in the congregation, the use of the means of grace—the Word and the Sacraments—is the most significant and, under all historical conditions, necessary for the preservation and growth of the congregation; for which reason also those gifts of grace that qualify for the administration of the means of grace, and quite especially for the proclamation of God’s Word, are those which the congregation esteems most highly and cultivates with special oversight.

§ 6. The activity of the Spirit toward the inner and outer growth of the congregation thus takes place through the gifts of grace and the means of grace; and since order and coherence are required in the congregation’s work within its own midst and in and for the surrounding world, there follows therefrom, of necessity, the organization of the congregation, in order that the work may become as effective as possible and that all members of the congregation may find their place therein.

§ 7. The relation of the means of grace to the congregation consists in this: that through the Word faith in Jesus Christ is created and preserved; through Baptism the believers are received into a communion of life with Jesus Christ, and thereby into His body, the congregation; and through the Lord’s Supper this communion of life between the Lord and the congregation is maintained and preserved.

§ 8. The relation of the congregation to the means of grace consists in this: that the congregation administers them through men whom it elects thereto; in this election the congregation seeks to find those men whom the Lord Himself has equipped with the gifts that are especially required for the proclamation of the Word and the spiritual guidance of the congregation. The inward call to this work thus consists essentially in the gift of grace from God, whereas the outward call consists in the congregation’s election to labor in the specific field. Since no difference of rank takes place among the members of the congregation, neither can the congregation’s election or ordination bring about any such difference, inasmuch as the matter concerns only the distribution of the work among the persons set thereto. The less, therefore, Scripture ascribes to the pastors any rule over the congregation, the more it holds true that all in the congregation are bound to the obedience of faith toward the Lord’s Word.

§ 9. From these outward conditions of the congregation it follows that from the one first congregation there arise many congregations, which nevertheless all together constitute only the one Trinitarian communion on earth that began with the sending of the Holy Spirit; which communion is therefore also in the New Testament called by the same name as the individual congregation (ekklesia, namely), whereas we frequently give it the name Church (from kyrialke, an adjective with the implied ekklesia, meaning “belonging to the Lord”). This new name, however, must not lead us to the thought that thereby there should be designated a communion which stood above the congregation or possessed a nature different from it.

§ 10. This Christian communion on earth is, despite its division into many congregations, despite the various and in part mutually conflicting confessions, despite the many different ecclesiastical organizations, despite the theological controversies, nevertheless only one; for the unity consists (invisibly) in the one Spirit and the one Lord, together with all the spiritual properties that follow therefrom, and (visibly) in the one and the same Word of God, the one Baptism, the one Supper of the Lord, the common ecumenical confessions, and a multiplicity of lesser and less significant distinguishing marks. The pietistic division of which the history of the Church bears witness is that which occurred at the Reformation, when the opposition to the worldliness of the Church—an opposition which had been present with greater or lesser strength throughout the church history of antiquity and the Middle Ages—became so strong that it brought about a rupture which can only slowly be healed. Yet even this does not abolish the unity of the Church, since the Reformation itself has its roots precisely in the Christian foundation that was present within the Roman Church itself, albeit buried beneath the rubble of many human ordinances.

§ 11. The one Christian communion on earth is holy by reason of the indwelling and activity of the Holy Spirit within it, whereby the believing disciples of Jesus are sanctified in an ever more inward union with the Lord Jesus Christ, while those who have fallen away from the faith are, through the working of the Word and the Spirit upon their obstinate hearts, ripened for the final and unavoidable judgment. The holiness of the Church is therefore not annulled by the participation of hypocrites in the outward congregation.

§ 12. In the Apostolic Creed the Christians also confess faith in the catholicity of the Church, which consists in this, that the Holy Spirit works for the salvation and eternal blessedness of all human beings, and that this missionary activity does not cease until all who are willing to let themselves be saved have been saved, and thus the work of the Gospel and of the Spirit on earth has reached its perfect completion, in that the fallen human race has been restored from its fall, and those who, despite the offer of the Gospel, have resisted the Spirit and grace are excluded from the kingdom of God’s eternal glory and from the redeemed human race.

§ 13. The apostolicity of the Church, or the Christian conviction that the congregation remains the same as that which, in the period of New Testament revelation, was built upon the foundation of the apostles and prophets, rests upon the truth that the coming into being of the congregation, and the whole of its existence and history, are wrought and borne by the Holy Spirit; and it is recognized by this, that the congregation preserves its agreement with the holy New Testament Scripture, which is the complete and reliable testimony to the nature and reality of the congregation from the very beginning.

§ 14. Since the Church came into being through the outpouring of the Spirit into the hearts of Jesus’ Jewish disciples, and since these received from Jesus the command to make all nations his disciples, it became unavoidable that already in apostolic times the question should be raised and decided whether the Church was only a form of the Jewish religious community modified by faith in the true Messiah, and whether therefore the Church was bound by the Law that had been given to the Jews, so that the Church could be recognized only by the observance of Jewish customs and ceremonies. And through the struggle which the resolution of this question required, there was secured for the Christian faith the great result that the Church is a new community, dependent solely upon faith in Jesus Christ and upon the new creation which consists in the Spirit’s communication of the new life, so that none of the signs and symbols, customs and ceremonies of the Jewish Law are binding upon the Church, nor, consequently, necessary for salvation. This glorious victory over all demands for the observance of the Law has for the Church in all ages the abiding significance that it is impossible to bind it by such regulations, laws, customs, ceremonies, and the like, as are devised by men, and to make these, or their observance, marks of the one true Church.

§ 15. The question whether the Kingdom of God or the Kingdom of Heaven is the same as the Church must, in the first place, be answered to the effect that the Church can be the Kingdom of God only upon the earth; and, in the second place, that even when we speak only of the Kingdom of God here upon earth or in the present world, the concept of the Kingdom of God is nevertheless wider than the concept of the Church, inasmuch as it includes not only the life of the Spirit in the hearts, which is the inmost essence of the Church, but also the preparatory and preserving divine government and guidance of the human race, and indeed also the manifold effects of divine revelation in the Gospel outside of and alongside the Church, in the life of the human intellect, in art, science, literature, forms of state, and the like; just as also the great secular transformations which take place in the world, and especially in the historical course of mankind through the coming of Christ, belong under the concept of the Kingdom of God. The relation between the Church and the Kingdom of God upon earth must therefore be determined thus, that the Church is the central life-stream in the Kingdom of God upon earth, alongside of which there are perceived divine workings of Spirit and power which indeed belong to the Kingdom of God, but nevertheless do not serve for the salvation of those who are affected by them, unless they become means to draw them to the Son and to his body, which is the Church. The course of the Church is brought to its close and fulfillment by the return of Christ and the final judgment, when the Kingdom of Glory, with its rest and blessedness, takes the place of the Church’s labor, struggle, and tribulation.
