\subsection{The Worth of the Mission}

“The little one shall become a thousand, and the small one a mighty nation; I, the LORD, will hasten it in its time.” Isa. 60:22.

There are many who think that the mission to the heathen goes forward but slowly. And they make use of this as an excuse for their own sloth and indifference. Properly, this very circumstance—that it goes slowly—ought rather to have the effect upon them that they labored with so much the greater zeal; but they take it in an altogether different manner and suppose that, when it goes so slowly, the whole work is of no consequence; it never truly advances in any case. At times some of these secret opponents of the mission deal quite foolishly with their contemplation of its slow progress and assert that it is simply an impossibility that the heathen should be converted; for, say they, there are born each year many more heathen than are baptized from among them; therefore the number of heathen increases far more rapidly than the number of heathen Christians, and thus it goes according to the old rule, that when he who goes before moves faster than he who follows after, the latter cannot overtake the former.

This objection against the mission to the heathen is at present very much in fashion and does great harm by dulling and blunting spiritual interest. It is therefore salutary to consider it and to see how much it truly amounts to.

Viewed from one side, a Christian will at once perceive that it is altogether wrong to listen to such talk. For so long as there is even one heathen who turns to God and has his soul saved through faith in Jesus Christ, so long is there yield enough from the mission to the heathen, even if it were to cost far more in human lives and money than it now costs. Or is not a soul worth more than “the whole world”? If God’s Son gave His life for us, why should not we give our life for Him?

But it is not the intent and promise of the mission merely that now and then a single heathen soul should be saved through the preaching of the Gospel. It is truly promised that the Kingdom of God shall be spread from sea to sea, wherever men dwell, and that the peoples, the nations, shall be received into it and find salvation and peace. All readers of the Bible know that such are the promises, and such also is the Lord’s command concerning the work which He desires to have carried out by His disciples. Therefore there lies in this manner of speech a temptation of unbelief: It avails nothing to engage in mission, for the heathen peoples increase more rapidly than the congregation. For the Lord has promised that the Gospel of the Kingdom shall be preached unto all nations, and that the Gentiles shall glorify God for His mercy’s sake. If anyone therefore says that it is impossible to Christianize the peoples, he makes God a liar and denies the faithfulness of His promise.

The work of missions therefore aims not merely at gaining a soul saved here or there, but at gathering congregations of converted Gentiles who carry on the same ministry with the Word and the Sacraments as is now common in the so‑called Christian lands. And it is this latter aim which unbelief holds to be impossible according to statistical calculations.

But God’s way of reckoning is not as the way of men. When the scoffers come with their reckoning of time and say, Where is the promise of His coming? then He answers: But, beloved, be not ignorant of this one thing, that one day is with the Lord as a thousand years, and a thousand years as one day. — And when the scoffers come with their statistics and censuses and say, It goes too slowly; the Kingdom of God will never reach unto the ends of the earth; then the Lord answers: The little one shall become a thousand, and the small one a mighty nation; I, the Lord, will hasten it in its time.

In other words, human calculations and human statistics do not accord with the Lord’s way of advancing His work. At times it sounds to us as though the Lord’s way and manner were precisely the opposite of our own.

Take, for example, Jesus’ manner of working in His last days.

When His following began to grow, and many joined themselves to Him; when He began to become “popular” through mighty works and authoritative power of His words; when the crowds grew great and would have made Him king—then He not only withdrew and vanished from the enthusiastic multitude, but He even returned to the people and spoke “hard words” to them, so that they turned away from Him and would follow Him no longer. And so strong did this movement away from Jesus become that He even asked the Twelve: Will ye also go away?—and yet He was fully resolved upon this, to “draw all unto Himself.” And in order to accomplish this, He thus drove all from Him, in that He went so deeply into shame and suffering that none were with Him and none dared to follow Him; “they all forsook Him, and fled,” says Mark. But Jesus went into humiliation and suffering and death alone, that He might become the grain of wheat which falls into the earth and dies, that it may bear much fruit. And when He then rose from the dead and ascended into heaven and sent the Spirit upon the disciples, then the multitude grew in wondrous wise according to the Lord’s good pleasure, and the prophetic word was fulfilled which says: The little one shall become a thousand, and the small one a mighty nation; I the LORD will hasten it in its time.

This is the Lord’s way. He does not always increase the strength of His host by adding; often He multiplies its power by taking away. Who does not remember Gideon’s army of two and thirty thousand men, of whom the LORD said: The people that are with thee are too many? And they were not few enough until they were but a mere three hundred men; then the victory was theirs.

This manner of the Lord’s working has revealed itself again and again in the history of missions. He lets His servants labor long; He wearies them; He lets them die upon the mission field without seeing visible results by human measure. But when the seed has vanished into the earth and been moistened with tears, at times with blood, then it sprouts; and one day there comes shoot upon shoot and blade upon blade, and the whole field grows green at once, so that it is a wonder before our eyes of life and nourishment.

It is not one mission, but it is well-nigh all missions, that have this experience: that the first beginning was small, poor, almost despairing. It seemed impossible to reach the hearts of the people. They would hear with the ears, none—or almost none—with the heart. Year after year passed; no change showed itself. Hope was awakened and again extinguished. When then all patience was on the point of breaking, the turning came: one soul sought the Lord until it found Him as its Savior; the first heathen was baptized. And from that hour, things changed. One came, and again another; many, and still more, came and sought salvation in that atonement which is in Jesus Christ. And the little one became a thousand, and the small one a mighty nation.

Concerning the Livingstone Mission in Central Africa it is related that it required six years of labor before a single heathen was baptized. When it was nine years old, it had nine baptized, believing members of the congregation. Now it is twenty-seven years old, and more than fifteen hundred have been baptized, and there are thousands in instruction for baptism.

And from other mission fields there come reports of the same kind. How then can it be said that it is impossible for Christianity to overcome heathenism? If it goes slowly in the beginning, it goes all the faster once it has gained momentum. Those who reckon that more heathens are born than are baptized—what good it their reckoning when the fire of revival begins to burn, as now in Madagascar, and the Lord burns up both heathenism and the reckonings?

There is only the one thing that matters: faith. It holds fast to the divine promise; it goes straight forward with its gaze fixed upon the shining star of the promise, and whether one predicts for it good or ill, it is steadfast in its purpose. When the Lord’s hour comes, then it goes in every place and among every heathen people as He Himself has said: The little one shall become a thousand, and the small one a mighty nation; I, the Lord, will hasten it in its time.

“Basleren,” 1902, pp. 97–100.

\subsection{A Year of Grace from the Lord}

The Spirit of the Lord GOD is upon me; because the LORD hath anointed me to preach good tidings unto the meek; he hath sent me to bind up the brokenhearted, to proclaim liberty to the captives, and the opening of the prison to them that are bound; to proclaim the acceptable year of the LORD, and the day of vengeance of our God; to comfort all that mourn. Isa. 61:1–2.

There cannot be the Spirit of God in a human being without there arising life and labor—most especially that labor which is portrayed in the two foregoing verses of the sixty-first chapter of the prophet Isaiah. And this, indeed, is precisely what is called missionary work.

But now it is also thus: no human being can be a Christian, and no society or fellowship can be a congregation, unless it has the Spirit of God. For no one is a Christian who does not have faith; yet no one has faith unless it is wrought by the Holy Spirit; and no fellowship is the Body of Christ, which is the congregation, unless by the Holy Spirit it is united with the Head, who is Christ.

If, therefore, anyone confesses himself to be a Christian, or if a gathering of people confesses itself to be a congregation, then let them reflect upon their calling and their task. The prophet Isaiah here paints it gloriously, both for the individual and for the congregation.

Begin with the beginning! In all things this is important, but most of all in Christian matters. What is required is to be able to say in truth: “The Spirit of the Lord GOD is upon me, because the LORD hath anointed me.”

Have we the Spirit of the Lord GOD? Are we anointed of the Lord? For only he has the Spirit of the Lord who is anointed of the Lord. In this respect it is true: of thyself thou canst take nothing; all thou canst receive from God. And again: it is not of him that willeth, nor of him that runneth, but of God that showeth mercy.

Therefore it is not for us a question of new resolutions and new promises on New Year’s Day. It is a question of the Spirit of God and the power of God. And we cannot put this question away from us by saying that we cannot reach so high or grasp so far, that it is impossible to attain unto the Spirit of God. For the Spirit of God is given freely and without price from God. Yea, more than this: the Spirit of God works upon our hearts in order to enter in, knocks at the door of the heart to have it opened. We can receive the Spirit of God if we open to him and give him room.

If we have used Christmas and the Christmas Gospel rightly, then it should be so with us now at the New Year that we could in truth say: “The Spirit of the Lord GOD is upon me, because the LORD hath anointed me.” For no other means are required in order to receive the Spirit from God than precisely that blessed Gospel of the Savior who is born in Bethlehem. There is Spirit to be received at Christ’s manger. Hast thou been with the shepherds in Bethlehem and seen “this thing which is come to pass”? Hast thou, with the aged Simeon in the temple, “taken the child into thine arms”? Then thou hast also experienced what “the unspeakable gift of God” is, and thou hast been laid hold of by God’s love toward poor human beings, and hast learned to praise the Lord, who in his Son has given us eternal life.

But then thou hast received the Spirit from God. Then thou art anointed of the Lord. For this anointing is precisely the love of God poured out into our hearts by the Holy Spirit, who is given unto us.

Hast thou then truly this Anointing from God? Doth this Fire burn in thy heart? Hast thou received a coal of fire upon thy tongue, so that thou canst praise God and speak of all His great works?

Perhaps thou sayest, No. And perhaps thou addest, that thou hast no thought of becoming either preacher or missionary, and therefore hast no need of such a gift of the Spirit? But thou art mistaken, if thou thinkest this to be any excuse, as though the Spirit of God were appointed only for some Christians and not for others. Thou must have the Spirit of God, if thou wilt be saved. Thou must have the Spirit, if thou wilt belong to the Lord. Thou must have the Spirit, if thou wilt be blessed and receive the crown of life. Thou knowest indeed that there are differences of gifts of grace, but the Spirit is the same. All Christians must have the Spirit; for “as many as are led by the Spirit of God, they are the sons of God.” The Spirit is the pledge of our redemption, and only they who have the pledge shall be able to attain the eternal redemption.

If therefore there be any who still have not the Spirit of the Lord of lords, for him the true New Year hath not yet begun. He is not fit to perform the work that belongeth to the anointed. He can receive no better counsel than this: Make haste to get oil in thy lamp! Thou mayest perhaps hold it in thy hand, but it shineth not; nor can it be made to burn, though thou shake it and apply a spark to it; there is no oil. Therefore make haste to go to the Lord in the Word, that thou mayest receive of His Spirit; then shalt thou have a light unto thy path and a lamp unto thy foot, that thou thyself mayest walk in the light; and then shalt thou also be able to lift the light on high, that it may become for others a benefit and a joy.

First and above all, therefore, we pray for this: that all our readers will sincerely prove themselves, whether they indeed are new men, led by the Spirit of God, who walk in the light and have fellowship one with another, and not with one another only, but also with the Father and the Son. Then shall we also experience this, that the year shall become unto us a new year of grace from the Lord, and from day to day the blood of Jesus Christ, the Son of God, shall cleanse us from all sin.

Thus comes the Call. For the Spirit is a spiritual driving power which unceasingly urges forward the cause of the Kingdom of God and takes into His service all those whose hearts He fills with the love of God. If we are driven by the Spirit of God, then the love of Christ constrains us to go where the Lord sends us, and to perform what He commands us.

And if we are to sum up all that the Lord commands us in a single word, then it is this: Mission. For this, at least for our part, is what we mean by Mission, as Isaiah says: “to preach good tidings unto the meek; to bind up the brokenhearted; to proclaim liberty to the captives, and the opening of the prison to them that are bound; to proclaim the acceptable year of the Lord, and the day of vengeance of our God; to comfort all that mourn.”

Yes, you will say, but this is the work of Jesus that is here described! Yes, it is; but are not the followers of Jesus to do the works of Jesus? Most certainly; He says it Himself: Verily, verily, I say unto you, He that believeth on Me, the works that I do shall he do also; and greater works than these shall he do; because I go unto My Father. And whatsoever ye shall ask in My Name, that will I do, that the Father may be glorified in the Son.

There is a real and living connection between Jesus and His believers. He sits at the right hand of the Father in glory, and His disciples are here below on earth in the Church militant; but the Spirit ascends, and the Spirit of God is sent down, and with the Spirit there follows light and power to do the work of God and of our Lord Jesus Christ upon the earth.

The New Year which we celebrate reminds us in a special way of this one thing which in truth comprehends all: “to proclaim the acceptable year of the Lord.” This means first and foremost that now is the time to preach the Gospel upon the earth; and so long as the Gospel is preached, and as often as it is preached, it shall sound with the voice of a mighty trumpet among the peoples: Behold, now is the accepted time; behold, now is the day of salvation! Yes, precisely now; for the Gospel is preached precisely when God wills, and precisely where God wills. And He wills nothing else thereby than to save sinners and to make them citizens of the Kingdom of Heaven.

Therefore we dare so boldly to proclaim the Year of Grace also this year. For still there is time, and still there is room for all who hear the Gospel to be saved and made blessed, to come home to the Father’s house, to become guests at the great Supper and at the wedding table of the King’s Son.

And precisely now the opportunity is so good and the doors so open, that it would be a fearful responsibility not to make use of the opportunity. Already missionaries are able to go forth, already the languages of the peoples are known and the Bible translated into their mother tongues. Truly it is high time that we awake and go forward with might against Satan’s strongholds and the fortresses of darkness. The Lord calls upon His host and commands it to go forth into the battle for the Kingdom of Light and for the cause of peace. What else are we Christians for, if not to follow the Lord’s call and to spread His Kingdom and His Gospel? Is not this our calling, is not this our task? At home and abroad it is the salvation of souls, eternal life, glory with God, which is the goal of every Christian’s life and labor.

Dear brothers and sisters! Pray the Lord for the Spirit, that He may in truth come upon us to drive us to proclaim the Year of Grace for poor heathen. They sit in bonds and imprisonment, under the oppressive yoke of heathendom; it is a blessed work to proclaim liberty to these captives of death and the opening of the prison to those bound by Satan. Therefore be with us with renewed earnestness, with greater steadfastness and perseverance! Let no day in the new year pass by without thought of the mission, without prayer for it! If we truly become a missionary people by the mighty assistance of God’s Spirit, not by the strength of our own resolutions, then we ourselves shall be blessed by the Lord, and He Himself will set us to be a blessing upon the earth.

Our life is not long, and the fleeting years remind us of the accounting we must soon render, from time into eternity. Would it not be good to make use of the few fleeting days we have for labor for the eternal Kingdom which is never shaken, the Kingdom of God and of our Lord Jesus Christ.

“Basleren,” 1902, pp. 1–5.

\subsection{The way of the Lord}

\begin{quote}
And all the trees of the field shall know that I the LORD have brought down the high tree, have exalted the low tree, have dried up the green tree, and have made the dry tree to flourish: I the LORD have spoken and have done it. Ezekiel 17:24.
\end{quote}

This word of Ezekiel is a prophecy concerning the way of the LORD with the Messiah and the salvation that is in Him.

The Messiah was “a tender shoot from the topmost twig of the lofty cedar,” as Ezekiel expresses it. The lofty cedar is a designation of the house of David, and its uppermost twig is an image of the last descendants of the royal line. This tender shoot, says Ezekiel, the LORD planted upon the high mountain of Israel; and it bore branches and brought forth fruit and became a glorious cedar, in which the birds of heaven found their dwelling.

It came to pass contrary to all expectation. The LORD brought down the high and exalted the low; for emperors fell, and kingdoms were overthrown; yet the lowly man from Nazareth, the despised shoot from the stem of Jesse, founded the kingdom which stands for ever, and of which there shall be no end. Thus the LORD willed it to be; for that which is highly esteemed among men is abomination in the sight of the LORD. He takes pleasure in choosing that which is nothing, in order to put to shame that which is something.

Therefore He whom scoffers called “the carpenter’s son” has received a greater kingdom than any other, and from day to day the kingdom advances and makes men blessed children of God.

And as it was with the Messiah, so it is with His people. As with Jesus, so with His Church. She too was as an scorned offshoot and a stumbling stone in the eyes of the world, when she stood forsaken by all in a hostile world that surrounded her on every side. Yet the LORD, who had ascended into heaven from her, had not forsaken her. And when the world would trample down the Church, slay her with the sword, burn her with fire, destroy her at any cost, then this word of the LORD sounded in her midst:

\begin{quote}
Fear not, O Israel: for I have redeemed thee, I have called thee by thy name; thou art mine. When thou passest through the waters, I will be with thee; and through the rivers, they shall not overflow thee: when thou walkest through the fire, thou shalt not be burned; neither shall the flame kindle upon thee. For I am the LORD thy God, the Holy One of Israel, thy Saviour.
\end{quote}

Once again it has proved true that the Lord maketh the high tree low and the low tree high; for the great and mighty societies of men have fallen, but the Lord has spread abroad and exalted His lowly congregation, so that from Jerusalem it has come even unto the uttermost ends of the earth.

And as with the congregation, so also with the individual. In the case of the poor, wretched sinner who is saved by grace alone, it is shown most gloriously that the Lord is He who maketh the dry tree green, while He maketh a green tree dry. For many a false Pharisee, who had no need of a physician, has become the prey of death, while sinners and publicans have received life through the faith of the Son of God.

Verily, God is a wondrous God, and His judgments are unsearchable, and His ways past finding out!

When therefore we ask whether the Lord has any use for us in the service of His kingdom, in its victorious course over the earth, we ought to note this, both for our comfort and for our humiliation, that this is a question not of whether we are great enough, but whether we are small enough; not of whether we are rich enough, but whether we are poor enough. For the Lord cannot use others as His true servants and instruments than those who are so incapable in themselves that they will receive the sufficiency which is of God, which He gives by His Spirit. Only when the Lord Himself is allowed to fashion His instrument does the instrument become fit for His work. True it is that the Lord also uses the rebellious and proud powers of the world in His service; but it is against their will, and therefore they are used as bondservants, and “the bondservant abideth not in the house forever.” No, the Lord’s true handmaiden is His little congregation, His small believing children. They are small enough that the Lord can use them; were they great, they would glory in themselves and exalt their own name instead of the Lord’s.

This wondrous way of the Lord also gives us hope for our mission and for the heathen among whom the Lord has set us to labor. Verily, in our corner of the vineyard there grows only a low tree. So dead and dry is the heathen people to whom the Lord has sent us. Is it therefore in vain to labor? Yes, if we do our own work with our own power, then it is indeed beforehand certain that we labor in vain and can accomplish nothing. But the arm of the Lord is not shortened, and His hands have not become powerless. He can still make “a dry tree green,” and He will do it. Let us only “longing after the Lord, abide in the Lord our Savior.” For He is still the same and still walks the same wondrous way.

Only patience on our part! Listen not to the many unbelieving voices that sound around us, and perhaps even within us: There is nothing to be done in South Madagascar; the climate is so deadly, the people stand so low, it is of no use. Indeed it is of use; for the Lord will make “a low tree high,” that His name may be honored and made known among the peoples. Already we have seen the great works of the Lord on our mission field, and the Lord will yet further reveal Himself gloriously. Let us continue in prayer and faith, and the Lord shall yet create life out of the dead and build His congregation also in the darkest regions of Madagascar, to which He has sent us with the light of the Gospel.

— “Basleren,” 1902, pp. 129–131







