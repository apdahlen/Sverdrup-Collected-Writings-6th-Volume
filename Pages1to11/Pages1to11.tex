
\begin{center}
\includegraphics[width=0.9\textwidth]{OpenImage1.png}
\end{center}

\bigskip

%\section{First Section}

\section{Missionary Meditations}

The following Missionary Meditations were in their time published in the Lutheran Free Church’s mission periodical, “Gaaseren,” whose first editor was Professor Sverdrup. It lay greatly upon his heart to lay a good foundation for missionary interest and missionary labor within the home congregation. Therefore he built, with unfailing faithfulness, upon the Word of God. It may therefore with certainty be said that these meditations contain much true missionary wisdom. For this reason, and in the hope that they may prove a welcome addition to the not overly large selection of sound missionary reading both for home and congregation, they are now issued together. In this edition they have been arranged according to the sequence in which the texts employed are found in the Bible.—Ed.

\subsection{Be thou a Blessing!}

When the Lord said to Abraham: Get thee out of thy country, and from thy kindred, and from thy father’s house, unto the land that I will shew thee! And I will make of thee a great nation, and I will bless thee, and make thy name great; and be thou a blessing! Gen. 12:1–2.

Abraham is the father of believers, and his faith and walk are for us all an example upon which we are to fix our eyes. All that is said concerning Abraham’s calling and obedience therefore has its application to all Christians.

In truth there are only two ways in which a human being can live. And according as he lives in the one way or in the other, his life becomes either a curse or a blessing. It depends upon the heart, from which life proceeds. As Jesus says: A good tree cannot bring forth evil fruit, neither can a corrupt tree bring forth good fruit. Either make the tree good, and its fruit good; or make the tree corrupt, and its fruit corrupt.

Now there is, of course, no earnest man or woman who desires to live a life unto curse for himself and for others. When we set ourselves to lay out a plan for our life, we surely do not resolve to live unto curse. And yet there are many, many people who deceive themselves at this point. They live heedlessly according to their own desire and inclination, which Scripture calls to live after the flesh; and they do not consider that they “reap corruption of the flesh,” both for themselves and for others. They do not take the matter seriously, but imagine that they can live a self-seeking, worldly, and vain life, and then by a few alms to the poor or some gifts to the mission still become a blessing to other people, even though they themselves should become a curse.

But it cannot be so; there is no true blessing in such a life. A worldly and fleshly person cannot, by a few so‑called “good works,” restore anything of the harm he does among men by his worldly manner of life and his corrupted disposition. With such people it goes as Jesus says of the Pharisees: they are like whited sepulchres, which outwardly indeed appear beautiful, but within are full of dead men’s bones and of all uncleanness; or they are “like the graves which appear not, and the men that walk over them are not aware of them.”

Therefore it will not do to preach among men and say: Be a blessing! be a blessing! if one does not at the same time show them how they themselves may receive blessing. First we must receive, and then we can give. Without this, all becomes nothing but preaching of the Law, without power to bring forth anything but bondage. For it is the power of the Law that it “gendereth to bondage.” But the son of the bondwoman shall not be heir of the promise unto blessing for the nations; no, the son of the freewoman is the heir; in him shall all the families of the earth be blessed.

It is not altogether free from error that there is a certain misdirection in our work for the heathen mission at this point. There is so much talk of mission offerings, and relatively so little talk of a missionary mind. Hence there easily creeps in the false notion that in the heathen mission one may be a blessing with one’s money, even though one continues unto curse with one’s fleshly life. But if we are to “be a blessing” like Abraham, it is also necessary that we ourselves receive blessing from God. And this is the chief matter, because it is the beginning and the foundation, without which it is impossible to be a blessing.

How, then, can our heart and our life be blessed by God?

This is the question for us all; for only when we ourselves are blessed can we also be a blessing. Only when we become good can we do good; it avails nothing to demand good works from evil men.

We have said that Abraham is our pattern in his faith. And in Abraham’s life and walk we can see the true nature and manner of faith. We cannot be Abraham’s children and at the same time possess a faith that takes shape in a manner different from Abraham’s. But if, in Abraham, faith so ordered itself that he both lost and gained, both forfeited and received, both let go and was given—then in a like manner it must go with us, if we are to become true children of Abraham and thus a blessing among men.

Abraham had to leave his land and his kindred and his father’s house, and go to the land which the Lord would show him.

This is the true beginning for us all who would walk in the footsteps of Abraham and become real and genuine friends of missions. We must bid farewell to the world and to all that is our own. Without the break with the old, there is no room for the new being of faith. The old worldly and fleshly being consists in loving the world and loving oneself, in craving perishable and earthly things and seeking one’s own—one’s own honor, advantage, and pleasure. This has nothing to do with faith; for faith is a sure conviction of things not seen. Faith occupies itself with the invisible and eternal things. Therefore there is also no salvation or eternal life to be found in visible and earthly things; for the visible things are perishable, and the world passeth away with its lust.

If, then, anyone would belong to Christ and have a share in the kingdom of God and His righteousness, he must bid farewell to the world and to himself, leave all things, and seek salvation by grace alone for Jesus’ sake. All our boasting and all our glory, all our delight and all our merit are gone, when we come naked and poor, wretched and miserable, to the Cross, to receive the forgiveness of sins, and through the narrow gate enter upon the difficult way that leadeth unto life.

Thus we set out with Abraham toward the “unknown land,” and set our longing upon that which is above in heaven, not upon that which is on the earth. This is faith. And it is well-pleasing unto God when the heart no longer clings to anything earthly, but opens itself toward heaven and toward the spiritual things which the Gospel bears with it: righteousness, peace, and joy in the Holy Ghost. Yea, there is joy in heaven when a soul that was on the point of sinking into the cold waves of sin and death lays hold of the lifeboat from heaven in a living faith and dares all upon the Lord’s word.

He who thus believes is blessed together with the believing Abraham. And it is precisely this that matters. It is not the first thing to be a blessing; the first thing is to become blessed. All newly converted souls indeed have this in common with one another, that they are eager to be a blessing to others at once, without delay. And it is surely the Spirit of God who works this zeal and longing. Yet not all the newly converted take time to prove themselves, whether they have indeed received the blessing.

Pause therefore here for a little while. The Lord says to Abraham: I will bless thee—and be thou a blessing. Behold, this is the right order; do not run past it! Think of Peter, who said to the lame man at the gate of the temple: Silver and gold have I none; but such as I have give I thee. There are surely some who attempt to give what they do not have. Wait until thou hast received the blessing, before thou seekest to give it.

What then is the blessing that we must receive before we can be a blessing to others? What else can it be than the grace in Christ—the grace which grants forgiveness of sins and makes all things new? Then indeed the old is passed away, when all our sins are forgiven and forgotten, yea, cast into the depths of the sea by God’s merciful grace, and peace is bestowed upon us undisturbed. And then indeed all things have become new, when the love of God is shed abroad in our hearts by the Holy Ghost which is given unto us. For how wholly and entirely does not the love of God transform all things in and for a human being! How different does it appear in the heart when it is full of love; how different does it appear all around us when we behold men and life in the light of God’s love. Yea, truly all things are new then—whether it be our nearest ones, or our friends, or our enemies, our countrymen or distant heathen peoples; yea, all is so entirely transformed when God shows it to us in the light of His love.

Then also does a human being become a blessing to his fellow men. Then the light is kindled, and it shines for those who are in the house. And then it is no longer a servile bondage of forced works to spread light and peace and joy among men through God’s Gospel. Nay, it is a great joy of heart to be able to accomplish something for the cause of God’s kingdom on earth, so that ever more may come to taste the peace and the joy in Jesus Christ.

Be a blessing!—this is among the new commandments of which John speaks, which are not grievous, those that are given to the children, but not to the slaves. For God Himself gives the love; and if we truly love, then the commandment is already fulfilled. For he who loves, not with word only, neither with the tongue alone, but in deed and in truth, is already a blessing.

Let us follow in the footsteps of Abraham’s faith; then shall God bless us with true love, and we shall become a blessing also for the heathen peoples. For the mission to the heathen is a work of love, unto which God has called us and equipped us by His own Holy Ghost.

“Gaaseren,” 1901, pp. 194–197.

\subsection{The Firm Foundation}

\begin{quote}
In thy seed shall all the nations of the earth be blessed. Gen. 22:18.

Now to Abraham and his seed were the promises made. He saith not, And to seeds, as of many; but as of one, And to thy seed, which is Christ. Gal. 3:16.
\end{quote}

The mission rests upon a firm and unshakable foundation, upon God’s faithful promise, which was spoken to Abraham and his Seed. Deep is this foundation laid; for the promise that was given to Abraham is grounded in that counsel of salvation which God purposed within Himself before the foundation of the world was laid, and which He revealed in the fulness of time in His own Son, our Lord Jesus Christ.

Long has this foundation already been tested in the history of mankind, without failing or being moved. For the promise was given to Abraham in that most ancient time when the nations were divided, and the Lord suffered them to walk in their own ways after the stubbornness of their evil hearts. The nations turned away from the living God and made themselves idols; they changed the truth of God into a lie, and worshipped and served the creature more than the Creator, who is blessed for ever. Yet God forgot not the nations; He determined for them appointed times and the bounds of their habitation, that they should seek the Lord, if haply they might feel after Him and find Him. And He also gave the promise to Abraham, that it should, through his lineage, in the people of Israel, and in the fulness of time, meet the need of the Gentile nations, when in the day of their distress they cried unto their idols and received no answer.

Through the two thousand years from Abraham unto Christ the Lord’s promise stood as a star shining in the darkness upon Israel’s heaven. Storms passed by, and clouds obscured it; yet the star was the same, and when the clouds were gone, it shone again with the same calm and clear light. What the promise had spoken concerning blessing for the nations in Abraham’s Seed was repeated and unfolded by the prophets, and ever more strongly and clearly did the divine counsel step forth, the nearer it drew toward the day of its fulfillment.

And the promised Seed of Abraham, who should be for a blessing unto all nations, came—Jesus Christ. The blessed kingdom of grace was founded by His death and resurrection. The middle wall of partition, which stood barring the way between Jews and Gentiles, was broken down. The Name was sent forth; the Gospel, in the very mother tongue of the nations, was laid near to their hearts and called them unto repentance and faith.

Since then the kingdom of God has spread from people to people, and the promise to Abraham is fulfilled continually from day to day. It is God’s eternal thought of peace with mankind that is being realized through the work of His Spirit and of His Church in the Gospel. For these are the days in which “the Spirit and the Bride say, Come! and let him that heareth say, Come!” God’s Church and each individual believer are “God’s fellow workers” in bringing forward that glorious day when the Gospel shall sound from sea to sea, among all peoples, unto the ends of the earth.

In this manner the Seed of Abraham becomes a blessing unto all nations. For there is no greater gift that can be rendered to the nations than the Gospel, which is a savour of life unto life for every one that believeth, and the kingdom of God, which is righteousness, and peace, and joy in the Holy Ghost.

Also in our missionary work in Madagascar it is the chief matter to build with full confidence upon this tried, divine promise. It has stood its test now for nearly four thousand years, since the day it was given to Abraham. It has proved its truth despite all the objections of unbelief and the mockery of men. Christ’s suffering shall have its reward, and His resurrection from the dead shall prove its victorious power. We may labor in perfect security, if we build our missionary work upon the immovable rock of the promise.

It is therefore the first greeting which our missionary journal desires to bring to its readers: Believe the promise, hold fast to it! Let all labor for the salvation of the Gentiles rest upon it. Only thus does the work become a work of God, in which His children are His instruments, whom He employs in His service for the salvation of souls. And only thus may we be fully assured that the work shall succeed, since it stands fast: “In thy seed shall all the nations of the earth be blessed”—and this seed of Abraham is Jesus Christ, who in and through His Gospel comes to souls as the Bread from heaven that giveth life unto the world.

Brethren and Sisters! it is a great and glorious calling to have a share in the work of the fulfillment of God’s counsel of salvation in the world. God will take us into His service as His fellow workers, that the Gospel may reach unto the ends of the earth. Eternal life and blessedness, and a fullness of blessings that follow therewith, the Lord will through His Church spread abroad over the earth among the wandering Gentiles. Us He calls to be His instruments. Shall we not all be obedient to His call and take up the work He entrusts to us with strength and zeal?

The Lord grant that none of those who have experienced the power of Christ’s death upon their heart may stand idle and inactive, when the Lord calls to manly deed and vigorous labor!

“Gaaseren,” 1900, pp. 2–4.

\subsection{Seed With Tears}

They that sow in tears shall reap with shouts of joy. They go forth and weep, bearing the seed that they scatter; they shall come home with shouts of joy, bearing their sheaves.
Psalm 126:5–6.

These two verses of the Psalm lay down, simply and powerfully, the law for all spiritual life upon earth—most assuredly also for the work of missions. Yet we find it so difficult to appropriate this law that, although we hear it often and say it frequently—that seed sown in tears brings a harvest of joy—we are nevertheless each time equally discouraged and fainthearted when sorrow and distress press tears from our eyes.

It ought not to be so. Or does the farmer lose heart when the heavens darken over the newly sown field? When the clouds gather and stand like a threatening wall, when the storm breaks loose and sweeps on mighty wings across the fields? When the rain drips from heaven, first drop by drop, then thicker and thicker, until it pours down like a flood? Is it then that the farmer trembles with fear and thinks that all his labor has been in vain, that nothing but famine awaits him and his household? Ah no. He may indeed keep himself sheltered on the day of the storm, but through the window he gazes out, and his heart quivers with joy when he sees that the seed is being thoroughly soaked by the rain of heaven. Then he says with David: “Thou hast visited the earth and watered it; thou greatly enrichest it; the river of God is full of water: thou preparest them corn, when thou hast so provided for it. Thou waterest the ridges thereof abundantly, thou settlest the furrows thereof: thou makest it soft with showers, thou blessest the springing thereof.”

As rain in the spring, so are sorrow and tears in the work of God’s Kingdom. Therefore, just as the farmer fears the clear sky and the unceasing sunshine once the seed has come into the ground, so also those who labor in God’s Kingdom ought to fear it when the beginning is bright and clear and everything succeeds at the first word. Deep must the ground be broken, and sore must suffering be endured, before golden fruit crowns the labor in God’s Kingdom.

All spiritual life among us, the whole blessed harvest of God’s Kingdom upon the great, broad field of the people and within the small, poor hearts of men, is altogether formed from the seed of pain which Jesus laid down through His suffering and death. All salvation for souls is a fruit of the heavy sufferings of Jesus Christ. As the psalmist says: “What Thou hast ploughed with such heaviness has become for us a harvest of joy.” It looked dark indeed, and it was immeasurable pain for Jesus when He was to become the grain of wheat that must be cast into the earth and die. “With strong crying and tears,” Scripture says, He offered up prayers and humble supplications in Gethsemane unto Him who was able to save Him from death. “And being in an agony He prayed more earnestly: and His sweat was as it were great drops of blood falling down to the ground.” In such torment and tears Jesus became the seed-corn that was to bear life and blessedness for a lost race of mankind. But because He poured out His soul unto death and held nothing back from any suffering, therefore God has given Him glory and exaltation and made Him a Savior for a multitude which no man can number, who shall eternally praise His Name, because He suffered and strove and conquered in the anguish of death.

And as it is with Jesus and the salvation that is in Him—that the great harvest of joy has come forth from the great sowing of suffering—so it is also with each and every one of us who now praise God for salvation and life. We did not win salvation in bliss and gladness. Experience shows far more the truth of that word which the Lord speaks: Woe unto you that laugh now, for ye shall mourn and weep. From the sowing of laughter and bliss there grows bitter sorrow and everlasting torment.

Neither did we win salvation by labor and toil, by vigorous struggle and glittering crowns of victory. It availed nothing unto salvation that much work was done and much was renounced and even much was prayed, so long as our own power was not broken and our own strength not shattered. The Christian life does not begin with the victory of the natural powers, but with their total defeat. And it is a day of weeping and sorrow when a strong man is bowed to his knees by sorrow for sin and by consciousness of guilt. Yet it is a good day, because it is the springtime—the day of spiritual spring—on which the living seed of God’s Word sinks deep into the soil of the heart and finds a well-prepared place where it may grow.

Far otherwise are the evil days for a man when all seems light and joy, when fortune smiles and wealth increases, and the smile of self-satisfaction spreads across a fat face. These are the days of spiritual drought, when spiritual life withers away and disappears, and a man ripens for the fire.

There must be spiritual humility of heart in order to receive the help of God’s grace and love; there must be brokenness in order to experience God’s healing. Therefore every Christian has a thorough experience of this precious word: They that sow in tears shall reap with shouts of joy.

But notwithstanding that we have these things so clear and incontrovertible to our consciousness; notwithstanding that we see it so plainly, both in the whole laying of the foundation of salvation through Jesus, and in the individual soul’s experience of salvation, that only those who begin with weeping can complete their course with joy—yet it is nevertheless so difficult to reconcile ourselves to this, that the Law is the same also for all true labour in the Kingdom of God. It must begin with birth-pain and travail, if it is to attain unto the power of life and the joy that follows therefrom.

This applies not least to the work of Gentile missions. For it is nothing other than to lay the grain of wheat, Jesus Christ, into new soil, where it had not previously been sown. It is nothing other than that new life is created in the hearts of lost sinners through the Gospel of Jesus Christ. Can this take place without pain? Can it happen without a rain of tears?

No, of that there can be no question. Suffering must come; both the Christians who bear the Gospel out to the Gentiles, and the Gentiles who receive the Gospel among themselves, must suffer distress and tribulation, if the Kingdom of God is truly to be planted in new ground. He who preaches Christ among the Gentiles must suffer; for he is a warrior who advances against the devil’s strong fortresses. The Gentiles among whom the Kingdom of God is planted must suffer; for their strength is broken and their dominion taken from them, when the idols are overthrown and Christ is enthroned among them.

When Paul was called to be apostle to the Gentiles, the Lord says of him: “He is to me a chosen vessel, to bear my Name before Gentiles and kings and the children of Israel; for I will show him how much he must suffer for my Name’s sake.” And was it not abundantly fulfilled in Paul, that which was spoken concerning suffering? Hear how he himself bears witness thereto, when, as he says, he “speaks in foolishness”:

\begin{quote}
I have laboured more abundantly, received more stripes, been imprisoned more often, been oft in peril of death. Of the Jews five times received I forty stripes save one. Thrice was I beaten with rods, once was I stoned; thrice I suffered shipwreck; a night and a day have I been in the deep. In journeyings often, in perils of waters, in perils of robbers, in perils from my own countrymen, in perils in the wilderness, in perils in the sea, in perils among false brethren; in labour and toil, often in watchings, in hunger and thirst, often in fastings, in cold and nakedness; beside those things that are without, that which cometh upon me daily, the pressure of concern for all the congregations.
\end{quote}

And the Gentiles among whom the Gospel was first spread—what shall we say of them and of their suffering? These bleeding congregations on the one hand, how unspeakably must they not have suffered year after year, century after century? If the Church was founded by the death of the Saviour, then surely its roots were watered with the blood of martyrs. Bloody was the dawn of the congregation, and tear-bedewed were its tender shoots. No wonder that it grew quickly and flourished well and spread far and wide.

—And on the other hand, who can portray or paint the sufferings of those who set themselves against the Gospel? The Roman Empire and its heathenism sank into ruin under the most violent and dreadful tribulations. It is not possible to defy the Lord and His goodness without judgment coming; for where the carcase is, there will the vultures be gathered together.

And in the most recent times we have the example of China. The world studies with horror the scenes which China’s latest mission history sets before us. Blood and tears moisten page after page of this history. Missionaries and native Christians—tortured, abused, beaten to death—until their number and their agonies call to mind the terrors of the ancient persecutions of the Christians. And heathen Chinese—slaughtered, burned in their houses, beheaded, driven to suicide; thousands and yet thousands of them have become the prey of death because of the conflict between darkness and light.

But just as the Roman Empire was permeated by the Gospel through many and long tribulations, so it shall now go with the Chinese Empire. Its walls are burst and fallen, and the Gospel of Christ can no longer be kept out. The seed is already growing in many places, and the harvest will come richly blessed through the tribulations. Much yet remains to be suffered; but the outcome is sure; for they that sow in tears shall reap with shouts of joy.

It is therefore right to hope for good when one suffers evil in the work of God’s kingdom. And what is said of all such work applies also to our little mission work in Madagascar. It has had its tribulations, and many tears have flowed also over this work. Sorrow and distress have followed it and follow it still. But that is no reason to lose heart. It must have its springtime with showers and storms over the seed. We must grieve over our losses, but we must be bold in our work and in our tribulations; for it is the Lord’s way.

Now we go and weep, bearing the seed which we scatter; let us be faithful therein. That is the right way to sow. Only that which is sown in this manner will bear good fruit. Therefore, while we suffer, let us be glad in hope; for the day comes when he that sowed with tears shall come home with shouts of joy, bearing his sheaves.

“Gaaseren,” 1902, pp. 33–37.

